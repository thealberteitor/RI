%%%%%%%%%%%%%%%%%%%%%%%%%%%%%%%%%%%%%%%%%
% Short Sectioned Assignment LaTeX Template Version 1.0 (5/5/12)
% This template has been downloaded from: http://www.LaTeXTemplates.com
% Original author:  Frits Wenneker (http://www.howtotex.com)
% License: CC BY-NC-SA 3.0 (http://creativecommons.org/licenses/by-nc-sa/3.0/)
%%%%%%%%%%%%%%%%%%%%%%%%%%%%%%%%%%%%%%%%%

%----------------------------------------------------------------------------------------
%	PACKAGES AND OTHER DOCUMENT CONFIGURATIONS
%----------------------------------------------------------------------------------------

\documentclass[paper=a4, fontsize=11pt]{scrartcl} % A4 paper and 11pt font size

% ---- Entrada y salida de texto -----

\usepackage[T1]{fontenc} % Use 8-bit encoding that has 256 glyphs
\usepackage[utf8]{inputenc}
%\usepackage{fourier} % Use the Adobe Utopia font for the document - comment this line to return to the LaTeX default

% ---- Idioma --------

\usepackage[spanish, es-tabla]{babel} % Selecciona el español para palabras introducidas automáticamente, p.ej. "septiembre" en la fecha y especifica que se use la palabra Tabla en vez de Cuadro

% ---- Otros paquetes ----

\usepackage{url} % ,href} %para incluir URLs e hipervínculos dentro del texto (aunque hay que instalar href)
\usepackage{amsmath,amsfonts,amsthm} % Math packages
%\usepackage{graphics,graphicx, floatrow} %para incluir imágenes y notas en las imágenes
\usepackage{graphics,graphicx, float} %para incluir imágenes y colocarlas
\usepackage{listings}
\usepackage{subfig}

% Para hacer tablas comlejas
%\usepackage{multirow}
%\usepackage{threeparttable}

%\usepackage{sectsty} % Allows customizing section commands
%\allsectionsfont{\centering \normalfont\scshape} % Make all sections centered, the default font and small caps

\usepackage{fancyhdr} % Custom headers and footers
\pagestyle{fancyplain} % Makes all pages in the document conform to the custom headers and footers
\usepackage{eurosym} % Para poder añadir el símbolo del euro
\fancyhead{} % No page header - if you want one, create it in the same way as the footers below
\fancyfoot[L]{} % Empty left footer
\fancyfoot[C]{} % Empty center footer
\fancyfoot[R]{\thepage} % Page numbering for right footer
\renewcommand{\headrulewidth}{0pt} % Remove header underlines
\renewcommand{\footrulewidth}{0pt} % Remove footer underlines
\setlength{\headheight}{13.6pt} % Customize the height of the header

\numberwithin{equation}{section} % Number equations within sections (i.e. 1.1, 1.2, 2.1, 2.2 instead of 1, 2, 3, 4)
%\numberwithin{figure}{section} % Number figures within sections (i.e. 1.1, 1.2, 2.1, 2.2 instead of 1, 2, 3, 4)
%\numberwithin{table}{section} % Number tables within sections (i.e. 1.1, 1.2, 2.1, 2.2 instead of 1, 2, 3, 4)

\setlength\parindent{0pt} % Removes all indentation from paragraphs - comment this line for an assignment with lots of text

\newcommand{\horrule}[1]{\rule{\linewidth}{#1}} % Create horizontal rule command with 1 argument of height

% Margins
\usepackage[margin=1.25in]{geometry}

% Begin section numbering at 0
\setcounter{section}{-1} 

% Hyperlinks
\usepackage{hyperref, xcolor}
\hypersetup{
  % hidelinks = true,   % Oculta todos los enlaces.
  colorlinks = true,   % Muestra todos los enlaces, sin bordes alrededor.
  linkcolor={black},     % Color de enlaces genéricos
  citecolor={black!40!blue},   % Color de enlaces de referencias
  urlcolor={black!40!blue}     % Color de enlaces de URL
}

\usepackage{listings}
%----------------------------------------------------------------------------------------
%	TÍTULO Y DATOS DEL ALUMNO
%----------------------------------------------------------------------------------------

\title{	
\normalfont \normalsize 
\textsc{\textbf{Recuperación de Información (2019-2020)} \\ Doble Grado en Ingeniería Informática y Matemáticas \\ Universidad de Granada} \\ [25pt] % Your university, school and/or department name(s)
\horrule{0.5pt} \\[0.4cm] % Thin top horizontal rule
\huge Preprocesado de documentos \\ % The assignment title
\horrule{2pt} \\[0.5cm] % Thick bottom horizontal rule
}
\author{Simón López Vico \\ Miguel Cantarero López \\ Alberto Jesús Durán López} % Nombre y apellidos
\date{\normalsize\today} % Incluye la fecha actual


\lstset{
	language=java,
	breaklines=true,
	basicstyle=\footnotesize,
	frame=single
}


%----------------------------------------------------------------------------------------
% DOCUMENTO
%----------------------------------------------------------------------------------------

\begin{document}

\maketitle % Muestra el Título

\newpage %inserta un salto de página

\begin{enumerate}
\item \textbf{Sobre los documentos (libros del proyecto Gutenberg) utilizados en la práctica anterior, hacer un estudio estadístico sobre los distintos tokens que se obtienen al realizar distintos tipos de análisis ya predefinidos. Por tanto, será necesario contar el número de términos de indexación así como frecuencias
de los mismos en cada documento. Realizar un análisis comparativo entre los distintos resultados obtenidos.}

Para este ejercicio hemos desarrollado una función la cual, pasado un \textit{analyzer} como parámetro, divide el texto en sus diferentes \textit{tokens} y guarda la frecuencia de cada uno de ellos en un \texttt{map}. Finalmente, ordena las palabras obtenidas por el número de ocurrencias en el texto (de mayor a menor) y muestra las 20 primeras junto al número total de \textit{tokens}.


\begin{lstlisting}
public static void Analyzer(Analyzer analyzer, String name, String contenido) throws IOException{
  System.out.println(name);
  TokenStream stream = analyzer.tokenStream(null, contenido);
  CharTermAttribute attr=stream.addAttribute(CharTermAttribute.class);
  stream.reset();
  
  Map<String,Integer> palabra_ocurrencias = new HashMap<String, Integer>();
  
  while(stream.incrementToken()){
    String palabra=attr.toString();
    //System.out.println(palabra);
    Integer value=palabra_ocurrencias.get(palabra);
    if(value==null) palabra_ocurrencias.put(palabra,1);
    else            palabra_ocurrencias.put(palabra,value+1);
  }
  
  palabra_ocurrencias=sortByValue(palabra_ocurrencias);		
  PrintPalabras();
  
  System.out.println("\nNumero de tokens en el fichero: " + palabra_ocurrencias.size() + "\n");
  stream.end();
  stream.close();
}
\end{lstlisting}

Con esto, podemos llamar a cualquier analizador incluido en \textit{Lucene} mediante el código \texttt{Analyzer(new WhitespaceAnalyzer(), "WhitespaceAnalyzer", contenido);}, donde \texttt{contenido} será el texto plano extraído con Tika del archivo parseado.

Hablemos ahora sobre los analizadores utilizados y sus respectivas salidas\footnote{Para hacer la comparación entre los distintos analizadores utilizados, nos centraremos en el fichero \texttt{odisea-es-txt.txt}.}:
\begin{itemize}
	\item \textbf{WhitespaceAnalyzer}: un analizador muy simple, solo usa el \textit{WhitespaceTokenizer}, el cuál separa el texto por los espacios en blanco. Como resultado obtenemos:
	
	\texttt{*************** WhitespaceAnalyzer ***************\\de - 8120		á - 6957		y - 6817		que - 5440		la - 5039		[...]\\	Número de tokens en el fichero: 25769}\\
	
	\item \textbf{StandardAnalyzer}: utiliza el \textit{StandardTokenizer}, además de pasar todos los tokens generados a minúscula y elimina las palabras que se encuentran en \texttt{ENGLISH\_STOP\_WORDS\_SET}. Este analizador puede reconocer URLs. El resultado obtenido es el siguiente:
	
	\texttt{*************** StandardAnalyzer ***************\\de - 8257		y - 7547		á - 7045		que - 5640		la - 5150		[...]\\Número de tokens en el fichero: 16194}
	
	Como vemos, con este analizador encuentra muchos menos tokens que con el WhitespaceAnalyzer, lo que es normal, pues para WhitespaceAnalyzer ``de'' y ``De'' serían dos tokens diferentes, mientras que para StandardAnalyzer es el mismo token.\\
	
	\item \textbf{SimpleAnalyzer}: consiste en la aplicación de \textit{LetterTokenizer} y  \textit{LowerCaseFilter}, es decir, define los tokens como cadenas máximas de letras adyacentes y luego pasa estos a minúscula. A diferencia de StandardAnalyzer, este analizador no puede reconocer URLs. Obtenemos lo siguiente:
	
	\texttt{*************** SimpleAnalyzer ***************\\de - 8259		y - 7548		á - 7045		que - 5640		la - 5151		[...]\\Número de tokens en el fichero: 15419}\\
	
	\item \textbf{StopAnalyzer}: hace lo mismo que SimpleAnalyzer, pero además elimina el conjunto de palabras que le pasemos (\textit{StopWords}) mediante el filtro \textit{StopFilter}. Hemos ejecutado usando un conjunto de palabras vacío y usando el conjunto de palabras inútiles en español, obteniendo los siguientes resultados:
	
	\texttt{*************** StopAnalyzer EMPTY\_WORDS\_SET ***************\\de - 8259		y - 7548		á - 7045		que - 5640		la - 5151		[...]\\Número de tokens en el fichero: 15419\\
	*************** StopAnalyzer SPANISH\_STOP\_WORDS\_SET ***************\\ulises - 1706		telémaco - 745		así - 547		pues - 525		si - 512		[...]\\Número de tokens en el fichero: 15171}
	
	Para el caso del conjunto vacío obtenemos el mismo resultado que SimpleAnalyzer, lo que era de esperar; por otra parte, eliminando las palabras inútiles en español comenzamos a tener una información más relevante sobre el tema del que trata nuestro texto.\\
	
	\item \textbf{SpanishAnalyzer}: este analizador es el que más operaciones realiza sobre el texto y los tokens obtenidos de él. Tras pasar los filtros estándar, pasa el \textit{SpanishPossessiveFilter}, que eliminará los ``restos'' posesivos de las distintas palabras. Después pasará todo a minúscula y eliminará las \textit{stopwords}, para finalmente utilizar el filtro \textit{PorterStemFilter}. Obtenemos el siguiente resultado:
	
	\texttt{*************** SpanishAnalyzer ***************\\á - 7045		ulis - 1705		telemac - 744		dios - 608		así - 547		[...]\\Número de tokens en el fichero: 12747}\\
\end{itemize}

Con estos resultados, podemos concluir que una buena opción para obtener el tema del que trata un texto sería usar el StopAnalyzer aplicándole el \texttt{SPANISH\_STOP\_WORDS\_SET}, aunque también es una buena decisión usar SpanishAnalyzer (o algún tipo de LanguageAnalyzer dependiendo del idioma del texto), pues obtenemos muchos menos tokens que con el resto de analizadores, los cuáles son fácilmente interpretables.\\



\item \textbf{Probar sobre un texto relativamente pequeño el efecto que tienen los siguientes tokenFilters:}
\begin{itemize}
	\item StandardFilter: Lleva obsoleto desde la versión 7.5.0 de Lucene, y en la versión actual (8.2.0) ni si quiera aparece, por lo que no hemos podido ejecutarlo.
	
	\item LowerCaseFilter: Pasa todo los tokens a minúscula.
	
	\texttt{ayer me compré un camión negro y por la tarde me comí un filete de ternera}
	
	\item StopFilter: Elimina los tokens que se encuentra en el \texttt{array\_set} que le pasemos.
	
	\texttt{Ayer compré camión negro tarde comí filete ternera}
	
	\item SnowballFilter: Filtra las palabras según el idioma que le pasemos como parámetro mediante el método de \textit{bola de nieve}.
	
	\texttt{Ayer me compr un camion negr y por la tard me com un filet de terner}
	
	\item ShingleFilter: Este filtro crea combinaciones de tokens como un único token con una longitud dada.
	
	\texttt{Ayer - Ayer me - me - me compré - compré - compré un - un - un camión - camión - camión negro - negro - negro y - y - y por - ...}
	
	\item EdgeNGramTokenFilter: Acorta los tokens a una longitud dada.
	
	\texttt{Aye com cam neg por tar com fil ter}
	
	\item NGramTokenFilter: Acorta los tokens como en el anterior filtro, pero además crea nuevos tokens acortando los originales empezando en distintas posiciones de la palabra.
	
	\texttt{Aye yer com omp mpr pré cam ami mió ión neg egr gro por tar ard rde com omí fil ile let ete ter ern rne ner era}
		
	\item CommonGramsFilter: Combina tokens que aparecen con frecuencia, los cuales serán pasados como argumento. Aplicándolo a nuestro texto definiendo como términos de alta frecuencia las palabras ``camión'' y ``filete'' obtenemos:
	
	\texttt{Ayer me compré un un\_camión camión camión\_negro negro y por la tarde me comí un un\_filete filete filete\_de de ternera }
	
	\item SynonymFilter: Agrupa los distintos sinónimos en un único token. Para aplicar este filtro se necesita un mapa a de sinónimos, el cuál no nos proporciona Lucene y es complicado de implementar, por lo que no se ha ejecutado el filtro para comprobar su funcionamiento.\\
\end{itemize} 



\item \textbf{Diseñar un analizador propio. Se os da libertad para poder escoger el dominio de ejemplo que consideréis mas adecuado, justificar el comportamiento.}

Para este ejercicio, como se nos daba libertad para escoger nuestro dominio de ejemplo, hemos realizado 2 analizadores propios:

\begin{itemize}
	\item  El primero se trata de un analizador de nombres llamado \textit{NameAnalyzer}, donde hemos creado nuestra propia clase con el mismo nombre. Dado un archivo, se encargará de reconocer las palabras escritas en mayúsculas referentes a nombres propios, ciudades, etc, y borrar las palabras que vayan precedidas de un punto y no aporten información (artículos, determinantes, preposiciones...).

	Para ello, se ha realizado una función que \textit{parsea} un archivo usando Tika. Además, dependiendo del idioma de nuestro fichero, usará el archivo \texttt{es.txt} (español) o \texttt{en.txt} (inglés) que contendrá las palabras que no aportan información para eliminarlas.

	\item Para el segundo analizador propio, llamado \textit{MyAnalyzer}, hemos usado la clase \textit{CustomAnalyzer} la cuál facilitará la creación de un analizador a medida. Con ella, hemos establecido como \textit{tokenizer} el \textit{StandardTokenizer} y hemos añadido distintos filtros para poner todos los tokens en minúscula, darles la vuelta y poner su primera letra en mayúscula.

	\begin{lstlisting}
Analyzer myanalyzer = CustomAnalyzer.builder()
  .withTokenizer("standard")
  .addTokenFilter("lowercase")
  .addTokenFilter("reversestring")
  .addTokenFilter("capitalization")
  .build();
}
	\end{lstlisting}
	\vspace{0.4cm}
\end{itemize}


\item \textbf{Implementar un analizador específico que para un token dado se quede únicamente con los últimos 4 caracteres del mismo (si el token tiene menos de 4 caracteres es eliminado). Para ello, debemos de crear un TokenFilter y diseñar el comportamiento deseado y el método incrementToken.}

Para este ejercicio hemos implementado un TokenFilter (last4CharFilter) que realice la comprobación de la longitud del token que le llega del analizador y posteriormente, si es válido, se quede con los últimos 4 caracteres.

La clase \textit{last4CharFilter} hereda de la clase abstracta \textit{TokenFilter} y utiliza el constructor de ésta. Hemos implementado un nuevo método \texttt{accept()} que indica si el token tiene una longitud mayor o igual a 4 para saber si hay que eliminarlo. 

Además hemos sobrescrito el método \texttt{incrementToken()} tal y como se pide en el enunciado.En este método se comprueba si el token es válido, en cuyo caso se procede a el volcado en un nuevo buffer (\texttt{newBuffer}) de los 4 últimos caracteres, y se sustituye por el anterior buffer. Si no es válido se saltan las posiciones que ocupe el token y pasa al siguiente.

\begin{lstlisting}
@Override
public final boolean incrementToken() throws IOException {
  int skippedPositions=0;
  while(input.incrementToken()){
    if (accept()) {
      //si nos hemos saltado algun token no valido anteriormente avanzamos posiciones con el atributo posIncrAtt
      if (skippedPositions != 0) {
        posIncrAtt.setPositionIncrement(posIncrAtt.getPositionIncrement() + skippedPositions);
      }
      //realizamos la copia de los 4 ultimos caracteres del token a un nuevo buffer y lo copiamos en termAtt
      int length = termAtt.length();
      char[] buffer = termAtt.buffer();
      char[] newBuffer = new char[4];

      for (int i = 0; i <4 ; i++) {
        newBuffer[3-i] = buffer[length-1-i];
      }
      
      termAtt.setEmpty();
      termAtt.copyBuffer(newBuffer, 0, newBuffer.length);
      return true;
    }
    //guardamos las posiciones avanzadas
    skippedPositions += posIncrAtt.getPositionIncrement();
  }
  // reached EOS -- return false
  return false;
}
\end{lstlisting}

\end{enumerate}

\newpage

\section*{Trabajo en Grupo}
Para la realización de esta práctica cada integrante del grupo ha contribuido en cada uno de los ejercicios, aportando entre todos la información necesaria para el correcto funcionamiento de las actividades que se piden. Además, Simón López Vico ha profundizado en el ejercicio 1 y 2, Alberto Jesús Durán López ha profundizado en el ejercicio 3 y Miguel Cantarero López ha profundizado en el ejercicio 4.

Este informe se ha realizado entre los tres integrantes del grupo.


\end{document}

